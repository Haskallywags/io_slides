%!TEX program = xelatex
\documentclass{beamer}

\usepackage{listings}
\usepackage{color}
\usepackage{hyperref}
\usetheme{Execushares}

\title{IO Without Breaking a Sweat}
\subtitle{Explaining Haskell's IO without Monads}
\author{Chris Wilson}
\date{March 3, 2014}

\setcounter{showSlideNumbers}{1}

% code listings (other settings:
%  language, firstline, lastline)
\lstloadlanguages{Haskell}
\lstset{
  basicstyle=\ttfamily,
  language=Haskell,
  tabsize=4,
  upquote=true,
  xleftmargin=\parindent
}

\begin{document}
\setcounter{showProgressBar}{0}
\setcounter{showSlideNumbers}{0}

\frame{\titlepage}

\begin{frame}{Contents}
\begin{enumerate}
\item IO as a Value \\ \textcolor{ExecusharesGrey}{\footnotesize\hspace{1em} See how IO is just another value like anything else}
\item Working with IO  \\ \textcolor{ExecusharesGrey}{\footnotesize\hspace{1em} Use the fact that \emph{IO is a value} to get stuff done}
\item Next Steps \\ \textcolor{ExecusharesGrey}{\footnotesize\hspace{1em} Some thoughts and links}
\end{enumerate}
\end{frame}

\setcounter{framenumber}{0}
\setcounter{showProgressBar}{1}
\setcounter{showSlideNumbers}{1}


\section{Introduction}

\begin{frame}{Before I Start}
This presentation is based on Neil Mitchell's excellent blog post,
\emph{Haskell IO Without Monads}: \\
\small{\url{http://neilmitchell.blogspot.co.uk/2010/01/haskell-io-without-monads.html}}
\end{frame}


\section{IO as a Value}

\begin{frame}{Four IO Functions}
\begin{itemize}
\item \lstinline{readFile :: FilePath -> IO String} \\ read in a file
\item \lstinline{writeFile :: FilePath -> String -> IO ()} \\ write out a file
\item \lstinline{getArgs :: IO [String]} \\ get command line arguments (as a list of strings)
\item \lstinline{putStrLn :: String -> IO ()} \\ write a string to the
console followed by a newline
\end{itemize}
\end{frame}

\begin{frame}[containsverbatim]{Simple IO Pattern}
\begin{lstlisting}
main :: IO ()
main = do
   src <- readFile "file.in"
   writeFile "file.out" (operate src)

operate :: String -> String
operate = -- your code here
\end{lstlisting}

\begin{itemize}
\item The ``processor'' function, \lstinline{operate}, is just a normal function.
\end{itemize}
\end{frame}

\begin{frame}[containsverbatim]
\frametitle{Action List Pattern}
\begin{lstlisting}
main :: IO ()
main = do
    x1 <- expr1
    x2 <- expr2
    -- ...
    xN <- exprN
    return ()
\end{lstlisting}
\end{frame}

\begin{frame}[containsverbatim]
\frametitle{Action List Pattern}
\begin{lstlisting}
main :: IO ()
main = do
    [arg1,arg2] <- getArgs
    src <- readFile arg1
    res <- return (operate src)
    _ <- writeFile arg2 res
    return ()
\end{lstlisting}
\end{frame}

\begin{frame}
\frametitle{Simplifying IO}
You can simplify IO according to three rules:
\begin{enumerate}
\item \lstinline{_ <- x} can be written as \lstinline{x}.
\item If the second-to-last thing in a do block has no binding arrow (\lstinline{<-}) and is of type \lstinline{IO ()}, then you can leave off the \lstinline{return ()}.
\item \lstinline{x <- return y} can be re-written as
\lstinline{let x = y}
\end{enumerate}
\end{frame}

\begin{frame}[containsverbatim]
\frametitle{Simplifying IO}
\begin{lstlisting}
main :: IO ()
main = do
    [arg1,arg2] <- getArgs
    src <- readFile arg1
    res <- return (operate src)
    _ <- writeFile arg2 res
    return ()
\end{lstlisting}
\begin{itemize}
\item We can re-factor our code using these rules!
\end{itemize}
\end{frame}

\begin{frame}[containsverbatim]
\frametitle{Simplifying IO}
\begin{lstlisting}
main :: IO ()
main = do
    [arg1,arg2] <- getArgs
    src <- readFile arg1
    res <- return (operate src)
    writeFile arg2 res
    return ()
\end{lstlisting}
\begin{itemize}
\item Rule 1
\end{itemize}
\end{frame}

\begin{frame}[containsverbatim]{Simplifying IO}
\begin{lstlisting}
main :: IO ()
main = do
    [arg1,arg2] <- getArgs
    src <- readFile arg1
    res <- return (operate src)
    writeFile arg2 res
\end{lstlisting}
\begin{itemize}
\item Rule 2
\end{itemize}
\end{frame}

\begin{frame}[containsverbatim]{Simplifying IO}
\begin{lstlisting}
main :: IO ()
main = do
    [arg1,arg2] <- getArgs
    src <- readFile arg1
    let res = operate src
    writeFile arg2 res
\end{lstlisting}
\begin{itemize}
\item Rule 3
\end{itemize}
\end{frame}

\begin{frame}[containsverbatim]{Nested IO}
\begin{itemize}
\item We can also \emph{nest} IO actions...
\end{itemize}
\begin{lstlisting}
title :: String -> IO ()
title str = do
    putStrLn str
    putStrLn (replicate (length str) '-')
    putStrLn ""
\end{lstlisting}
\begin{itemize}
\item ...and then use it in \lstinline{main}.
\end{itemize}
\begin{lstlisting}
main :: IO ()
main = do
    title "Hello"
    title "Goodbye"
\end{lstlisting}
\end{frame}

\begin{frame}[containsverbatim]{Returning IO Values}
\begin{itemize}
\item We're not just limited to the \lstinline{IO ()} type, we can return values from IO
\item This function returns the first two command line args as a tuple:
\end{itemize}
\begin{lstlisting}
readArgs :: IO (String,String)
readArgs = do
    xs <- getArgs
    let x1 = if length xs > 0
              then xs !! 0 else "file.in"
    let x2 = if length xs > 1
              then xs !! 1 else "file.out"
    return (x1,x2)
\end{lstlisting}
\end{frame}

\begin{frame}[containsverbatim]{Returning IO Values}
\begin{itemize}
\item Now we can use \lstinline{readArgs} in \lstinline{main}:
\end{itemize}
\begin{lstlisting}
main :: IO ()
main = do
    (arg1,arg2) <- readArgs
    src <- readFile arg1
    let res = operate src
    writeFile arg2 res
\end{lstlisting}
\end{frame}

\begin{frame}[containsverbatim]{Optional IO}
\begin{itemize}
\item In any \emph{real} program, we need to \emph{optionally} run code in response to input:
\end{itemize}
\begin{lstlisting}
main :: IO ()
main = do
    xs <- getArgs
    if null xs then do
        putStrLn "You entered no arguments"
     else do
        putStrLn ("You entered " ++ show xs)
\end{lstlisting}
\end{frame}


\section{Working with IO}

\begin{frame}[containsverbatim]{Remember, IO is a Value}
\begin{itemize}
\item Recall that the \lstinline{title} function had type \lstinline{IO ()}
\item Which means it can be used as-is in a do block to run the action three times
\item That is, we \emph{don't} have to immediately execute
\lstinline{IO} actions
\end{itemize}
\begin{lstlisting}
main :: IO ()
main = do
    let x = title "Welcome"
    x
    x
    x
\end{lstlisting}
\end{frame}

\begin{frame}[containsverbatim]{We Can Pass Arguments to IO}
\begin{lstlisting}
replicateM_ :: Int -> IO () -> IO ()
replicateM_ n act = do
    if n == 0 then do
        return ()
     else do
        act
        replicateM_ (n-1) act
\end{lstlisting}
\begin{itemize}
\item We recursively run the \lstinline{IO ()} as many times as we need,
\item so, rewriting our last example:
\end{itemize}
\begin{lstlisting}
main :: IO ()
main = do
    let x = title "Welcome"
    replicateM_ 3 x
\end{lstlisting}
\end{frame}

\begin{frame}[containsverbatim]{Store IO in Structures}
\begin{itemize}
\item \lstinline{sequence_} runs a list of actions in turn:
\end{itemize}
\begin{lstlisting}
sequence_ :: [IO ()] -> IO ()
sequence_ xs = do
    if null xs then do
        return ()
     else do
        head xs
        sequence_ (tail xs)
\end{lstlisting}
\begin{itemize}
\item We can refactor \lstinline{replicateM_} using \lstinline{sequence_}:
\end{itemize}
\begin{lstlisting}
replicateM_ :: Int -> IO () -> IO ()
replicateM_ n act = sequence_ (replicate n act)
\end{lstlisting}
\end{frame}

\begin{frame}[containsverbatim]{Pattern Match}
\begin{itemize}
\item Keeping in mind that \lstinline{IO} is just a value, we can pattern match on it
\item let's refactor that definition of \lstinline{sequence_}:
\end{itemize}
\begin{lstlisting}
sequence_ :: [IO ()] -> IO ()
sequence_ []     = return ()
sequence_ (x:xs) = do
    x
    sequence_ xs
\end{lstlisting}
\end{frame}

\begin{frame}[containsverbatim]{A Short Example}
\begin{lstlisting}
main :: IO ()
main = do
    xs <- getArgs
    sequence_ (map operateFile xs)

operateFile :: FilePath -> IO ()
operateFile x = do
    src <- readFile x
    writeFile (x ++ ".out") (operate src)

operate :: String -> String
operate = -- ...your code here
\end{lstlisting}
\begin{itemize}
\item This performs \lstinline{operate} on each file given on the command line.
\end{itemize}
\end{frame}


\section{Next Steps}

\begin{frame}{Other Things to Check Out}
\begin{itemize}
\item \href{http://www.cs.nott.ac.uk/~gmh/book.html}{\emph{Programming in Haskell}} by Graham Hutton (chapters 8 \& 9)
\item \href{https://wiki.haskell.org/Monads_as_Containers}{Monads as Containers}
\item Many more useful functions can be found in the \href{http://haskell.org/ghc/docs/latest/html/libraries/base/Control-Monad.html}{Control.Monad} package
\end{itemize}
\end{frame}

\end{document}
